\documentclass[letterpaper,12pt]{scrartcl}
\usepackage[utf8]{inputenc}
\usepackage{amsmath}
\usepackage{amssymb}

\title{Amateur Radio Basic Qualification -- The Essentials}
\subtitle{Section One: Regulatory and Legal Requirements}
\author{University of Waterloo Amateur Radio Club}
\date{\today}

\begin{document}

\maketitle
\tableofcontents

\section{Introduction}

These notes were prepared from Issue 3 of RIC-7 ``Basic Qualification Question Bank for Amateur Radio Operator Certificate Examinations'', published April 2007.
They cover 100\% of testable material on the Basic Qualification examination, but do not go beyond what is absolutely necessary to know in order to pass the examination.
The candidate is encouraged to perform their own research on topics that are not fully covered here.

\section{The Essentials}

\subsection{Industry Canada}

\begin{itemize}
\item Industry Canada is the department that is responsible for the administration of the Radiocommunication Act.
\item The Radiocommunication Act grants authority to make ``Radiocommunication Regulations'' and ``Standards
for the Operation of Radio Stations in the Amateur Radio Service''.
\item The Radiocommunication Regulations defines the ``amateur radio service''. (Note that this is a \textit{different} document from the Radiocommunication Act.)
\end{itemize}

\subsection{Operator Certificates}

\begin{itemize}
\item There is no fee associated with obtaining an Amateur Radio Operator Certificate.
\item An Amateur Radio Operator Certificate is valid for life. (This is actually a relatively recent thing, effective April 1, 2000. Prior to that it was necessary to renew each year and there was an associated fee.)
\item Industry Canada requires address information from you when you obtain a radio licence.
\item Industry Canada must be notified of any change in postal address.
\item The Amateur Radio Operator Certificate should be retained at the address given to Industry Canada, and a copy must be retained at the station (if that is at a different address).
\item The holder of a radio authorization shall, at the request of a duly appointed radio inspector, show the radio authorization or a copy thereof to the inspector within 48 hours after the request.
\end{itemize}

\subsection{Breaking the Rules}

\begin{itemize}
\item Out of band transmissions are prohibited. You are only allowed to operate in designated amateur bands (frequency ranges).
\item The sending of false or deceptive signals is prohibited. This includes sending false emergency signals when there is no real emergency.
\item A person found guilty of transmitting a false or fraudulent distress signal, or interfering with or obstructing any radio communication without lawful cause,
may be liable, on summary conviction, to a penalty of a fine not exceeding \$5000, or a prison term of one year, or both.
\item No person shall decode an encrypted subscription programming signal without permission of the lawful distributor.
\item No person shall, without lawful cause, interfere with or obstruct any radiocommunication.
\item The Minister may suspend a radio authorization where the holder has contravened the Act, the Regulations, or the terms and conditions of the authorization.
This includes cases where the radio authorization was obtained through misrepresentation. 
\item If the holder of a radio authorization has failed to comply with a request to pay fees or interest due, the Minister may suspend their radio authorization.
This is the \textit{only case} where the Minister may suspend or revoke a radio authorization \textit{without notice}.
\item Radio inspectors are not above the law and require a warrant to enter dwellings without the consent of the occupant.
\end{itemize}

\subsection{Basic Qualification}

\begin{itemize}
\item There are no age limits with respect to eligibility to hold an Amateur Radio Operator Certificate.
\item The Basic examination is the first and only examination that must be passed before an Amateur Radio Operator Certificate is issued.
\item The holder of an ``Amateur Digital Radio Operator's Certificate'' has equivalency for the Basic and Advanced qualifications.
\item After obtaining the Basic qualification, additional qualifications may be earned in \textit{any order}.
\item Two Morse code qualifications are available: ``5 wpm'' and ``12 wpm''.
\item Holders of a Basic Qualification only cannot transmit on frequencies below 30~MHz (but more on this later).
\end{itemize}

\subsection{Higher Qualifications}

\begin{itemize}
\item Licensed radio amateurs may install, place in operation, repair, or maintain radio apparatus on behalf of another person
if the other person is the holder of a radio authorization to operate in the amateur service.
(If the other person \textit{doesn't} have any type of radio operator certificate, they are out of luck -- no one may operate on radio equipment on their behalf.)
\item Individuals with Advanced Qualification may build transmitting equipment for use in the amateur service.
(With Basic Qualification only, transmitting equipment must be either a commercial pre-assembled product or a packaged kit designated for amateur use.)
\item Holders of a Morse Code qualification, Basic Qualification with Honours, or Advanced Qualification may transmit on frequencies below 30~MHz.
\end{itemize}

\subsection{Licensing}

\begin{itemize}
\item Amateur stations must be licensed (i.e. operated by licensed operators) at all times. There are no exemptions in the amateur service.
\item Amateur stations may be used to communicate with other similarly licenced stations.
\item Radio amateurs may not transmit superfluous signals, or profane or obscene language or messages.
\item Radio amateurs may not operate or permit to be operated equipment that is not performing to the Radiocommunication Regulations.
\item Radio amateurs may not use equipment to amplify the output of licence-exempt transmitters.
\end{itemize}

\subsection{Communication}

\begin{itemize}
\item Discussion on amateur bands is limited to messages of a technical nature or personal remarks of relative unimportance.
\item Commercial or business use of amateur bands is prohibited.
\item Radio amateurs may not broadcast information to the general public.
\item False or deceptive amateur signals or communications may not be transmitted.
\item Secret codes or ciphers may not be used in the amateur service.
\item Procedural signals or ``prowords'' may be used in the amateur service if they do not obscure the meaning of a message.
\item Transmission of commercially recorded material on amateur bands is not allowed. It's recommended to turn down the volume of
music playing in the background if you are operating an amateur station.
\end{itemize}

\subsection{Location}

\begin{itemize}
\item The holder of an Amateur Radio Operator Certificate may operate an amateur radio station \textit{anywhere} in Canada.
\item A ``beacon'' station may transmit one-way communications.
\item Installation of a ``repeater'', or any other radio apparatus which automatically retransmits radio signals \textit{within the same frequency band},
can only be done by the holder of Basic and Advanced Qualifications.
\item Installation of a radio apparatus to be used specifically for an amateur radio club station can only be done by the holder of Basic and Advanced Qualifications.
\end{itemize}

\subsection{Responsibility}

\begin{itemize}
\item Both the control operator and the station licensee are responsible for the proper operation of an amateur station.
\item If you transmit from another amateur's station, both of you are responsible for proper operation.
\item Any qualified amateur chosen by the station owner may be the control operator of an amateur station.
\item An amateur station must have a control operator whenever the station is transmitting.
The control operator must be at the station's control point.
\item The owner of an amateur station may permit \textit{any person} to operate the station under the supervision and in the presence of the holder
of the operator certificate.
\end{itemize}

\subsection{Interference}

\begin{itemize}
\item A transmission that disturbs other communications is called ``harmful interference''.
\item You may never deliberately interfere with another station's communications.
\item Some amateur radio bands are designed ``Amateur Secondary''. Amateurs are allowed to use the frequency band only if they do not cause interference to primary users.
\item If two amateur stations want to use the same frequency, both station operators have an equal right to operate on the frequency.
\item Where interference to the reception of radiocommunications is caused by the operation of an amateur station,
the Minister may require that the necessary steps for the prevention of the interference be taken by the radio amateur.
\item Radio amateur operation must not cause interference to other radio services operating in the 430 to 450~MHz band and in the 902 to 928~MHz band.
(This is because these bands, among others, are designated ``Amateur Secondary''.)
\item The operator of an amateur station may conduct technical experiments, trials, or tests using station apparatus, provided that these experiments do not interfere with other stations.
\end{itemize}

\subsection{Exceptional Circumstances}

\begin{itemize}
\item In the amateur radio service, it is permissible to broadcast radio communications required for the immediate safety of life of individuals
or the immediate protection of property.
\item In an emergency situation, a large number of standard regulations cease to apply.
\item Amateur radio stations may communicate with \textit{any station} involved in a real or simulated emergency.
\item Business communications are still not permitted under any circumstances, even in an emergency.
\item If you hear an unanswered distress signal on an amateur band where you do not have privileges to communicate,
those privileges are temporarily granted and you should offer assistance if possible.
\item Similarly, an amateur radio station in distress may use \textit{any means} of radiocommunication, including transmissions in unqualified bands.
\item During a disaster, an amateur station may make transmissions necessary to meet essential communication needs and assist relief operations when
normal communication systems are overloaded, damaged, or disrupted.
\item In an emergency, \textit{there are no limitations} on the power output that can be used by a station in distress.
\item During a disaster, most communications are handled by ``emergency nets'' using predetermined frequencies in amateur bands.
Operators not directly involved with disaster communications are requested to avoid making unnecessary transmissions on or near
frequencies being used for disaster communications.
\item Messages from a recognized public service agency may be handled by amateur radio stations during peacetime and civil emergencies and exercises.
\item It is permissible to interfere with the working of another station if your station is directly involved with a distress situation.
\end{itemize}

\subsection{Message Passing}

\begin{itemize}
\item No payment of any kind is allowed for third-party messages sent by an amateur station.
(If it were allowed, it would count as a commercial service, which cannot be operated with an amateur licence.)
\item Radiocommunications transmitted by any amateur station may be divulged or used under any circumstances.
\item Radiocommunications, other than broadcasts or amateur transmissions, may be subject to penalties of they are divulged, intercepted, or used.
There are a few exceptions: where it is for the purpose of preserving or protecting proprty, or for the prevention of harm to a person;
where it is for the purpose of giving evidence in a criminal or civil proceeding in which persons are required to give evidence;
or where it is on behalf of Canada, for the purpose of international or national defence or security.
\end{itemize}

\subsection{Identification}

\begin{itemize}
\item Licenced amateur stations are given a ``callsign'' that is used to identify the licensee.
\item The call sign of a Canadian amateur radio station typically starts with the letters ``VA'', ``VE'', ``VO'', or ``VY''.
\item Amateur stations transmit their call sign to identify themselves on the air.
\item Amateur stations must identify themselves at the beginning and end of a contact, and at least every thirty minutes during communication.
(Each station transmits its own call sign at the beginning and end of communications.)
\item Unidentified communications are usually not allowed, with the exception of control signals for model crafts.
\item Either English or French may be used to identify a Canadian amateur station.
\end{itemize}

\subsection{International Regulations and Third-Party Traffic}

\begin{itemize}
\item If a non-amateur friend is using your station to talk to someone in Canada, and a foreign station breaks in to talk to your friend,
you need to find out if Canada has a third-party agreement with the foreign station's government to pass radio traffic
before communications can continue.
\item Radio amateurs may use their stations to transmit international communications on behalf of a third party only if
such communications have been authorized by the countries concerned.
\item The International Telecommunication Union regulates communication between countries, including international radiocommunication.
\item If a country has notified the ITU that it objects to international amateur communications, a person operating a Canadian amateur station is forbidden from contacting them.
\item Amateur third-party communications are transmissions of non-commercial or personal messages to or on behalf of a third party.
\item International communications on behalf of third parties may only take place if the countries concerned have authorized such communications.
\item Messages originated from the Canadian Forces Affiliated Radio Service (CFARS) and messages originated from the United States Military Affiliated Radio System (MARS)
do not count as ``communications on behalf of a third party''.
\item There exist a large number of countries with ``reciprocal operating arrangements'' with Canada. Essentially, what this means
is that licenced Canadian amateurs automatically receive qualifications to operate in those countries, and amateurs licenced in those countries
automatically receive qualifications to operate in Canada. One such country is the United States; it is \textit{not necessary} for US radio amateurs to obtain a Canadian amateur station licence
before operating in Canada.
\end{itemize}

\subsection{Privileged Abilities}

\begin{itemize}
\item When a station is used by someone other than the owner, the allowed operating privileges are defined by the largest set of privileges \textit{shared} by both the
station owner and the control operator. For example, if someone else with additional qualifications operates your station, only the privileges allowed by your qualifications are allowed.
\item In order to operate below 30~MHz, you must earn any Morse code qualification, or an Advanced qualification, or attain a mark of 80\% or higher on the Basic exam.
\item The licensee of an amateur station may operate radio controlled models on all frequencies above 30~MHz.
\end{itemize}

\subsection{Band Plan}

\begin{itemize}
\item The band plan dictates the range of frequencies that are available for use in the amateur service. Each frequency range, or ``band'', has different electromagnetic characteristics associated with it -- more on these later.
\item Bands may also be divided into ``sub-bands'' for specific purposes. For example, the 144--148~MHz band has a sub-band for voice communications and a sub-band for digital communications.
\item It is common practice to refer to bands by their wavelength instead of their frequency range. There is a simple equation that can be used to calculate the approximate operating frequency
given the wavelength of a band. The general law is $f = \dfrac{c}{\lambda}$, where $f$ is the frequency in Hertz, $c$ is the speed of light in meters per second, and $\lambda$ is the wavelength in meters.
If we wish to express the frequency in megahertz, we can write the constant $c$ as ``300 mega-meters per second''. Then, keeping the wavelength $\lambda$ in meters, take the approximate frequency in megahertz to be
$f \approx \dfrac{300}{\lambda}$. 
\item Now, having said that, the correct value for $c$ is slightly lower than this, so the value for $f$ over-estimates the true value. This is important because on the Basic exam
there are several questions which are of the form ``The $x$-metre amateur band corresponds to which of the following frequency ranges?'' and for very large or very small values of $x$ the fact that this formula
is an approximation can result in finding the wrong answer. So, if you don't feel that it's necessary for you to memorize the entire band plan and the frequency ranges for each named band,
remember that the answer from this formula is always an over-estimate. Take the next smallest answer to the frequency you calculate if you don't fall within one of the given ranges. If there are two answers
that would include the number you calculate, take the one that includes frequencies lower than your calculated $f$.
\item Let's do an example that you might see on the exam. What's the frequency range for the 20-meter band? Calculate $f \approx \dfrac{300}{20} = 15$. There are two answers that might match this:
one includes 15.000~MHz as the lower bound, and the other includes 14.350~MHz as the upper bound. These are the closest two answers, and the second one (14.000 to 14.350~MHz) turns out to be correct -- remember that $f$ \textit{over}approximates
the true frequency, so if your calculated $f$ falls on the edge of one of the answers, that answer is wrong.
\item There is one important exception to this rule regarding the ``15 meter band''. The frequency range for this band does not correspond to the given wavelength of the band, even when it is being
calculated correctly. If you use the formula here, you will make a mistake; this one needs to be memorized. The 15 meter band corresponds to the 21.000 to 21.450~MHz frequency range.
\end{itemize}

\subsubsection{Bandwidth}

\begin{itemize}
\item You will learn more about the technical nature of bandwidth in a later section. For now it suffices to know a few basic facts and figures that pertain to regulations.
\item The bandwidth of an amateur station is determined by measuring the frequency band occupied by the transmitted signal at a level of 26~dB below the maximum amplitude of that signal.
(A 26~dB reduction corresponds to an amplitude that is $\dfrac{1}{400}$ of the original value.)
\item Below 28~MHz, the maximum authorized bandwidth is 6~kHz, with the exception of the frequency range 10.1 to 10.15~MHz, where the maximum authorized bandwidth is 1~kHz.
\item The maximum authorized bandwidth in the frequency range of 28 to 29.7~MHz is 20~kHz.
\item The maximum authorized bandwidth in the frequency range of 50 to 148~MHz is 30~kHz.
\item Bandwidth generally increases with frequency -- consult the band plan for details.
\item You will need to know the bandwidth required to use several modes of operation (which has implications for which frequency ranges can be used for these modes).
\begin{itemize}
\item CW (Morse telegraphy) -- 150~Hz
\item AMTOR/RTTY -- 170 to 200~Hz
\item Packet radio -- varies depending on the speed of data transfer, but can be done with less than 1000~Hz of bandwidth
\item SSB -- 3000~Hz
\item Slow-scan television -- 3000~Hz
\item FM -- 5000 to 15000~Hz
\item Fast-scan television -- 6~MHz
\end{itemize}
\end{itemize}

\subsection{Transmit Power}

\begin{itemize}
\item Radio amateurs must use only the minimum legal transmitter power necessary to communicate.
\item Transceiver power is measured at the antenna terminals of the transmitter or amplifier. 
\item The holder of only a Basic qualification may use up to 250 watts of power, or 560 watts ``peak equivalent power'' (PEP) for single sideband operation.
\item With an Advanced qualification, up to 1000 watts of power may be used.
\end{itemize}

\subsection{Retransmission}

\begin{itemize}
\item A repeater station is an amateur station that automatically retransmits the signals of other stations.
\item An unmodulated carrier may be transmitted only for brief tests on frequencies below 30~MHz.
(Above 30~MHz, you run the risk of accidentally activating a repeater with your carrier, which is a bad idea.)
\item Radiotelephone signals in a frequency band below 29.5~MHz cannot be automatically retransmitted, unless these signals are received
from a station operated by a person qualified to transmit on frequencies below 29.5~MHz.
\end{itemize}

\subsection{Transmitter Control}

\begin{itemize}
\item When operating on frequencies below 148~MHz, the frequency stability must be comparable to crystal control.
(And above 148~MHz you're almost certainly using crystal control or something even better.)
\item A reliable means to prevent or indicate overmodulation must be employed at an amateur station if radiotelephony (voice) is used.
\item The maximum percentage of modulation permitted in the use of radiotelephony by an amateur station is 100 percent.
\item All amateur stations, regardless of the mode of transmission used, must be equipped with a reliable means of determining the operating radio frequency.
\end{itemize}

\subsection{ITU Regulations}

\begin{itemize}
\item In addition to complying with the Radiocommunication Act and Radiocommunication Regulations, Canadian radio amateurs must also comply with the 
regulations of the International Telecommunication Union (ITU).
\item Messages of a technical nature or personal remarks of relative unimportance may be transmitted to an amateur station in a foreign country.
\item It is forbidden to transmit international messages on behalf of third parties unless those countries make special arrangements.
\item Radiocommunications between countries shall be forbidden if the administration of one of the countries objects.
\item Administrations shall take such measures as they judge necessary to verify the operational and technical qualifications of amateurs.
\item The ITU Radio Regulations do not say anything about band restrictions for radio amateurs who have not demonstrated proficiency in Morse code.
(This is why a Morse code qualification is no longer necessary.)
\end{itemize}

\subsection{International Operation}

\begin{itemize}
\item Canada is located in ITU Region 2.
\item Australia, Japan, and Southeast Asia are in ITU Region 3.
\item A Canadian radio amateur operating a station in the territory of another country is subject to
frequency band limits applicable to radio amateurs licensed in that country.
For example, a Canadian operator using a station in the United States is subject to
whatever frequency band limits apply to US radio amateurs.
\item The European Conference of Postal and Telecommunications Administrators (CEPT) license allows for international operation in any of the 32 CEPT countries.
\item Canadian radio amateurs may apply for a CEPT licence in any CEPT country.
\item Canadian radio amateurs holding Basic and 12 wpm qualifications will be granted CEPT Class 1 recognition.
\item Canadian radio amateurs holding Basic qualification only will be granted CEPT Class 2 recognition (operation only above 30~MHz).
\item The reverse also applies -- foreign radio amateurs holding CEPT Class 1 licences will receive recognition in Canada equivalent to Basic and 12 wpm qualifications,
and those with Class 2 licences receive recognition equivalent to Basic qualification only.
\end{itemize}

\subsection{License Exams}

\begin{itemize}
\item The fee for taking an examination for an Amateur Radio Operator Certificate by an accredited volunteer examiner is to be negotiated.
\item The fee for taking the exam at an Industry Canada office is \$20 per qualification.
\item An accredited volunteer examiner must hold Basic, Advanced, and 12 wpm qualifications.
\item A disabled candidate taking a Morse code sending test may be allowed to recite the examination text in Morse code sounds.
\item Examinations for disabled candidates may be given orally or tailored to the candidate's ability to complete the examination.
\end{itemize}

\subsection{Antenna Structures}

\begin{itemize}
\item There is no requirement to receive prior approval from Industry Canada to construct an antenna or its support structure.
\item However, prior to an installation for which community concerns could be raised, radio amateurs must consult with their land-use authority, and possibly their neighbours.
\item For the purposes of environmental filing, amateur stations are considered to be Type 2 (non-site-specific).
\end{itemize}

\subsection{RF Exposure}

\begin{itemize}
\item Health Canada has published safety guidelines for the maximum limits of RF energy near the human body.
\item Safety Code 6 gives RF exposure limits for the human body.
\item According to Safety Code 6, frequencies in the range of 30 to 300~MHz cause the greatest risk from RF energy.
The limit of exposure to RF in this frequency range is the lowest because the human body absorbs RF energy the most in this range.
\item The maximum safe power output to the antenna of a VHF or UHF hand-held radio is not specified by Safety Code 6. The exemption
for portable equipment was withdrawn in 1999.
\item The maximum exposure levels of RF fields to the general population in the frequency range 10 to 300~MHz is 28 volts-RMS per meter.
(This is a measurement of electric field strength, or ``E-field''.) In the frequency range 30 to 300~MHz,
the maximum exposure level is 0.073 amperes-RMS per meter. (This is a measurement of \textit{magnetic} field strength, or ``H-field''.)
\item Permissible exposure levels of RF fields increase as frequency is increased above 300~MHz or below 10~MHz.
\end{itemize}

\subsection{Interference}

\begin{itemize}
\item ``Radio-sensitive equipment'' is defined as ``any device, machinery, or equipment other than radio apparatus, the use or functioning of which is or can be adversely affected by radiocommunication emissions''.
\item In the event of interference to third-party electronics systems, if the field strength of the amateur station signal is below 1.83 volts per meter,
it will be deemed that the affected equipment's lack of immunity is the cause. If the field strength is above this, it will be deemed that the transmission
is the cause of the problem.
\item Broadcast transmitters are not included in the list of field strength criteria for resolution of immunity complaints.
\end{itemize}

\end{document}

